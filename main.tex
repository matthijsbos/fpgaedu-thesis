\documentclass{uva-inf-bachelor-thesis}
\usepackage{graphicx}
\usepackage[hyphens]{url}
\usepackage{enumitem}
\usepackage{multicol}
\usepackage{csquotes}
\usepackage[margin=\parindent,format=hang]{caption}
\usepackage{subcaption}	
\usepackage{pdflscape}
\usepackage{float}
\usepackage{tocbibind}
\usepackage{listings}
\usepackage{xstring}
\usepackage{hyperref}
\usepackage{cleveref}
% \usepackage[a4paper]{geometry}
\usepackage{subfiles}
\usepackage{dirtree}

\usepackage{datetime}
\renewcommand{\dateseparator}{-}
\yyyymmdddate

\usepackage[style=ieee, backend=bibtex]{biblatex}
\addbibresource{cite.bib}

\usepackage{amsmath}
\usepackage{empheq}
\setlength{\fboxsep}{1em}

% \let\tmp\oddsidemargin
% \let\oddsidemargin\evensidemargin
% \let\evensidemargin\tmp
% \reversemarginpar


\lstset{captionpos=t, xleftmargin=\parindent, xrightmargin=\parindent, aboveskip=1em, belowskip=1em, abovecaptionskip=0em, belowcaptionskip=1em, escapeinside={(*@}{@*)}, float=h, frame=single, framexleftmargin=\parindent, framexrightmargin=\parindent, framextopmargin=1em, framexbottommargin=1em, basicstyle=\small}
% \BeforeBeginEnvironment{lstlisting}{\par\noindent\begin{minipage}{\linewidth}}
% \AfterEndEnvironment{lstlisting}{\end{minipage}}

% Title Page
\title{A Model for \\Experiment Setups on\\ FPGA Development Boards}
\author{Matthijs Bos}
\supervisors{A. van Inge \& T. Walstra, University of Amsterdam}
\signedby{}


\begin{document}

\maketitle

\begin{abstract}
This thesis proposes a model for the application of FPGA development boards as a tool for experimentation with digital logic. This model adopts the concept of an address space as a means for isolation of board-specific components in the logic's architecture, as well as a means for generalization of experiment setup interaction. This subsequently allows for the reuse of a number of architectural components across different projects as well as for the automation of different parts of the development process. As a result, the process for development of new FPGA configurations is shortened and reduced in terms of complexity.
\end{abstract}

\tableofcontents

% Chapters

\subfile{chapters/introduction}

\subfile{chapters/background}

\subfile{chapters/model}

\subfile{chapters/implementation}

\subfile{chapters/evaluation-analysis}

\subfile{chapters/conclusion}

% Appendices

% \appendix

% \subfile{chapters/process-analysis}

% \subfile{chapters/logic-diagrams}

% \subfile{chapters/experiment-setup-wrapper-sources}


\cleardoublepage

\restoregeometry

\addcontentsline{toc}{chapter}{Bibliography}
\phantomsection

\printbibliography

\end{document}